%!TEX program = xelatex
% 完整编译: xelatex -> bibtex -> xelatex -> xelatex
\documentclass[lang=cn,11pt,a4paper]{elegantpaper}

\title{地方补贴式租赁住房的挤出效应:来自LIHTC计划的新证据}
\author{任\;涛}
%\institute{\href{https://elegantlatex.org/}{Elegant\LaTeX{} 项目组}}

%\version{0.08}
\date{\zhtoday}

\usepackage{appendix}
\renewcommand{\appendixtocname}{附录}
\renewcommand{\appendixpagename}{附录}
\begin{document}

\maketitle

\begin{abstract}
  \hspace{2\ccwd}自1987年成立以来,低收入住房税收抵免(LIHTC)计划已迅速发展成为美国有史以来最大的低收入住房补贴建筑来源,占最近所有多户出租建筑的三分之一。本文研究了这种日益重要的中低收入住房来源的挤出效应。为此,我们分析了LIHTC建设对地理的三个不同层次(MSA,县和10英里半径圆)的影响。这使我们能够采用越来越广泛的地域固定效应,以帮助区别未观察到的因素。政治变量也被用作进一步促进识别的工具。
  
  \!在我们所有的模型中,IV估计产生的拥挤要比OLS大得多,这证实了LIHTC开发对新建筑成熟区域的内生吸引力。我们最可靠的IV估算表明,LIHTC发展的近100\%被 新建的无补贴租赁单位的数量,尽管这一点估计值附近的置信区间允许进行不太剧烈的评估。其他估算表明,LIHTC的发展对自住房建设的影响要小得多,但这些估算并不精确。总体而言,尽管LIHTC的发展可能会很好地影响低收入公寓的位置,但我们的估计表明,该计划对新开发的出租房屋数量的影响似乎很小。

  \keywords{挤出效应,保障性住房,LIHTC}
\end{abstract}
\vspace{10pt}

\begin{tcolorbox}[
	colback=red!5!white,
  colframe=red!30!black,
  fontupper = \itshape,
]
“I rise today to introduce the Affordable Housing Tax Credit Enhancement Act of 2005. … the bill would double the current LIHTC [annual allocations], which would yield twice the number of affordable units annually. … Today, the LIHTC program is widely regarded as the nation's most successful housing production program resulting in the construction and rehabilitation of more than 1.3 million housing units for lower income households. …”

\textbf{Statements Submitted to Congressional Record: May 26, 2005 By Rep. William Jefferson (D-LA)}
\vspace{5pt}

\tcblower

“我今天起草来介绍《2005年经济适用房税收抵免增强法案》。……该法案将使目前的LIHTC(年度拨款)翻一番,这将使每年的经济适用房数量增加一倍。 …今天,LIHTC计划被公认为是美国最成功的住房生产计划,它为低收入家庭建造和修复了130万套住房……”
\vspace{5pt}

\textbf{众议员William Jefferson(D-LA)在2005年5月26日提交国会记录的声明。}

\end{tcolorbox}

\section{引言}

提供给穷人的住房援助的方式仍然有很多甚至是激烈的争论:政府应该通过需求方优惠券类型计划(例如第8节优惠券)还是通过公共和低价等供应方建筑补贴在地方投资人 收入住房税收抵免(LIHTC)住房? 在此背景下,本文考察了快速增长的LIHTC计划,并强调了LIHTC建设在多大程度上排除了无偿租赁房屋的开发。 一些进一步的背景将有助于使LIHTC计划成为现实。

在1930年代末至1980年代中期,联邦政府通过“传统”公共住房计划建造了超过一百万套住房。 重要的是,这些计划通常将入住人数限制在贫困线以下或以下的家庭 \citep{Olsen2003365}\footnote{\cite{Olsen2003365}指出,至少有29种不同的公共住房计划。 这些项目中的家庭通常将其总收入的30\%用于租金。}。到1980年代,至少在两个方面,人们的担忧开始削弱对公共住房进一步扩大的支持。 首先是政府建设,拥有和运营公共住房项目。 关于某些活动是否最好留给私营部门存在一些基本问题。 第二个是公共住房项目造成了密集的贫困集群,加剧了人们对犯罪,邻里衰落以及对项目中儿童成长的不利影响的担忧(例如,可参见 \cite{Currie200099,Jencks1990111})。出于这些原因,公共住房的建设在1980年代初结束了。1990年代开始拆除性能最差的项目。\footnote{在某些情况下,例如根据HOPE VI计划,公共住房结构进行了改建,但在大多数情况下,通常向住户发放住房券,并告知他们私下寻求住房 \cite{Jacob2004233}。}

%\nocite{*}
\bibliographystyle{plainnat}
\bibliography{ref}

\appendix
\appendixpage
\addappheadtotoc
\section{How I became inspired}


\end{document}
